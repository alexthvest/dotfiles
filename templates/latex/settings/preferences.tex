% настройки polyglossia
\setdefaultlanguage{russian}
\setotherlanguage{english}

% локализация
\addto\captionsrussian{
	\renewcommand{\figurename}{Рис.}%
	\renewcommand{\partname}{Глава}
	\renewcommand{\contentsname}{\centerline{Содержание}}
}

% шрифты
\setmainfont{Times New Roman}
\newfontfamily\cyrillicfont{Times New Roman}

% перечень использованных источников
\addbibresource{refs.bib}

% настройки полей
\geometry{top=2cm}
\geometry{bottom=2cm}
\geometry{left=2cm}
\geometry{right=2cm}
\geometry{bindingoffset=0cm}

% настройка ссылок и метаданных документа
\hypersetup{unicode=true,colorlinks=true,linkcolor=red,citecolor=green,filecolor=magenta,urlcolor=cyan,
	pdftitle={\docname},
	pdfauthor={\student},
	pdfsubject={\subject},
	pdfcreator={\student},
	pdfkeywords={\subject}
}

% оформления подписи рисунка
\captionsetup[figure]{labelsep = period}

\DeclareCaptionFormat{hfillstart}{\hfill#1#2#3\par}
\captionsetup[table]{format=hfillstart,labelsep=newline,justification=centering,skip=-10pt,textfont=bf}

% путь к каталогу с рисунками
\graphicspath{{pictures/}}

% Внесение titlepage в учёт счётчика страниц
\makeatletter
\renewenvironment{titlepage} {
  \thispagestyle{empty}
}
\makeatother

\counterwithin{figure}{section}
\counterwithin{table}{section}

\titlelabel{\thetitle.\quad}
